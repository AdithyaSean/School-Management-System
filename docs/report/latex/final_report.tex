%%%%%%%%%%%%%%%%%%%%%%%%%%%%%%%%%%%%%%%%%%%%%%%%%%%%%%%%%%%%%%
% School Management System - Final Year Diploma Project Report
% LaTeX Template (for Overleaf or other compilers)
%%%%%%%%%%%%%%%%%%%%%%%%%%%%%%%%%%%%%%%%%%%%%%%%%%%%%%%%%%%%%%
\documentclass[12pt,a4paper]{report}
\usepackage[utf8]{inputenc}
\usepackage{graphicx}
\usepackage[a4paper, left=1.5in, right=1in, top=1in, bottom=1in]{geometry}
\usepackage{array}
\usepackage{enumitem}
\usepackage{titlesec}
\usepackage{amsmath}
\usepackage{amsfonts}
\usepackage{amssymb}
\usepackage{mathptmx} % Better Times New Roman implementation
\usepackage[T1]{fontenc} % Better font encoding
\usepackage{url} % For hyperlinking URLs
\usepackage{fancyhdr}
\usepackage{setspace}
\usepackage{hyperref}
\usepackage{longtable}
\usepackage{tocloft}
\usepackage{caption}
\usepackage{booktabs}

% Page Setup
\setstretch{1.5}
\pagestyle{fancy}
\fancyhf{}
\fancyfoot[C]{\thepage}

% Customizing section titles to match the document's style (simple and bold)
\titleformat{\chapter}[hang]{\bfseries\LARGE}{\thechapter}{2em}{}
\titleformat{\section}[hang]{\bfseries\large}{\thesection}{1em}{}
\titleformat{\subsection}[hang]{\bfseries}{\thesubsection}{1em}{}
\titleformat{\subsubsection}
  {\normalfont\normalsize\bfseries}{\thesubsubsection}{1em}{}

% Remove page number on the first page (cover page)
\pagenumbering{gobble}

\begin{document}

% --- Cover Page ---
\begin{titlepage}
    \centering
    \vspace*{0.3cm}
    
    \includegraphics[width=3.5cm]{nibm-logo.png} % NIBM Logo
    
    \vspace{1cm}
    
    {\Huge\bfseries NATIONAL INSTITUTE OF BUSINESS MANAGEMENT}
    
    \vspace{1cm}
    
    {\Large School of Computing and Engineering}
    
    \vspace{1.2cm}
    
    {\Huge\bfseries ``School Management System''}
    
    \vspace{1.2cm}
    
    {\Large\bfseries Final Year Diploma Project Report}
    
    {\large Academic Year 2022-23}
    
    \vspace{1.5cm}
    
    {\Large Submitted By:}
    
    \vspace{0.5cm}
    
    {\large E M A S B EKANAYAKA (CODSE224F-001)}
    
    {\large S D HEIYANTHUDUWA (CODSE224F-016)}
    
    {\large M H D T TISSERA (CODSE224F-054)}
    
    \vspace{1cm}
    
    {\Large Project Guide:}
    
    \vspace{0.3cm}
    
    {\Large\bfseries DR. THISARA WEERASINGHE}
    
    {\large Head of School of Computing and Engineering}
    
    \vfill % Pushes content to the bottom
    
    {\large Date: 2024/12/20}
    
    {\large Place: NIBM Colombo School of Computing and Engineering}
    
    \vspace{0.5cm}
    
    {\Large\bfseries Diploma in Software Engineering}
    
\end{titlepage}

\newpage
\pagenumbering{roman} % Start page numbering from here with Roman numerals

% --- Coursework Content ---

%%%%%%%%%%%%%%%%%%%%%%%%%%%%%%%%%%%%%%%%%%%%%%%%%%%%%%%%%%%%%%
% Declaration Page
%%%%%%%%%%%%%%%%%%%%%%%%%%%%%%%%%%%%%%%%%%%%%%%%%%%%%%%%%%%%%%
\section*{Declaration}
I/We hereby declare that this project report is my/our own work and has not been submitted previously for any academic qualification. All sources of information have been acknowledged.

\vspace{2cm}
\noindent\textbf{Signatures:}

\vspace{2cm}
\noindent\rule{6cm}{0.4pt}\\
\noindent E M A S B EKANAYAKA (CODSE224F-001)

\vspace{1cm}
\noindent\rule{6cm}{0.4pt}\\
\noindent S D HEIYANTHUDUWA (CODSE224F-016)

\vspace{1cm}
\noindent\rule{6cm}{0.4pt}\\
\noindent M H D T TISSERA (CODSE224F-054)

%%%%%%%%%%%%%%%%%%%%%%%%%%%%%%%%%%%%%%%%%%%%%%%%%%%%%%%%%%%%%%
% Abstract
%%%%%%%%%%%%%%%%%%%%%%%%%%%%%%%%%%%%%%%%%%%%%%%%%%%%%%%%%%%%%%
\newpage
\section*{Abstract}
The School Management System is a comprehensive software solution designed to address the growing needs of modern educational institutions. This project implements a robust platform that integrates various aspects of school administration, academic management, and communication into a unified system. The system employs cutting-edge technologies including biometric authentication, role-based access control, and real-time data processing to provide an efficient and secure environment for all stakeholders.

The implementation demonstrates significant improvements in administrative efficiency, reducing manual workload through automation of routine tasks and streamlining communication channels, thereby enhancing the overall educational experience. The system's modular architecture ensures scalability and maintainability, while its intuitive interface promotes rapid adoption among users of varying technical proficiency.

The project's significance lies in its potential to transform traditional educational management practices, particularly in government schools where resources are often limited. By digitizing core processes and providing real-time access to information, the system enables better decision-making and more effective resource utilization, ultimately contributing to improved educational outcomes.

\textbf{Keywords:} School Management, Education Technology, Academic Administration, Database Management, Web Application

%%%%%%%%%%%%%%%%%%%%%%%%%%%%%%%%%%%%%%%%%%%%%%%%%%%%%%%%%%%%%%
% List of Keywords, Figures, Tables, Acronyms, Acknowledgement
%%%%%%%%%%%%%%%%%%%%%%%%%%%%%%%%%%%%%%%%%%%%%%%%%%%%%%%%%%%%%%
\newpage
\section*{Acknowledgement}
We would like to express our sincere gratitude to our project guide, Dr. Thisara Weerasinghe, and the faculty of the School of Computing and Engineering for their invaluable support and guidance throughout this project.

%%%%%%%%%%%%%%%%%%%%%%%%%%%%%%%%%%%%%%%%%%%%%%%%%%%%%%%%%%%%%%
% Table of Contents
%%%%%%%%%%%%%%%%%%%%%%%%%%%%%%%%%%%%%%%%%%%%%%%%%%%%%%%%%%%%%%
\newpage
\tableofcontents

%%%%%%%%%%%%%%%%%%%%%%%%%%%%%%%%%%%%%%%%%%%%%%%%%%%%%%%%%%%%%%
% List of Figures and Tables
%%%%%%%%%%%%%%%%%%%%%%%%%%%%%%%%%%%%%%%%%%%%%%%%%%%%%%%%%%%%%%
\newpage
\listoffigures

\newpage
\listoftables

\newpage
\renewcommand{\thepage}{\arabic{page}} % Switch to Arabic numerals
\setcounter{page}{1}

%%%%%%%%%%%%%%%%%%%%%%%%%%%%%%%%%%%%%%%%%%%%%%%%%%%%%%%%%%%%%%
% Chapter 1: Introduction
%%%%%%%%%%%%%%%%%%%%%%%%%%%%%%%%%%%%%%%%%%%%%%%%%%%%%%%%%%%%%%
\chapter{Introduction}
\section{Background}
The digital revolution has transformed virtually every sector of society, yet many educational institutions, particularly government schools, continue to operate using traditional manual systems. This technological gap not only impacts administrative efficiency but also affects the quality of education and student engagement. In an era where students are increasingly tech-savvy and parents expect real-time updates about their children's progress, the need for a comprehensive digital solution has become paramount.

\section{Project Context}
This School Management System project emerges from a critical need to modernize educational institution operations, specifically targeting provincial schools such as Susamayawardhana College in Colombo 08. This school, which educates students from primary grades through to Advanced Level (A/L), represents a common scenario in the local education landscape where a comprehensive digital solution can make a significant impact. The initiative was conceived after extensive consultation with educators, administrators, and education technology experts, who identified significant opportunities for improving educational outcomes through digital transformation in such environments.

\section{Project Objectives}
\begin{itemize}
    \item Modernization of Educational Operations
    \item Improvement of Stakeholder Engagement
\end{itemize}

\section{Scope and Significance}
The project encompasses a complete overhaul of school management processes, from daily administrative tasks to long-term strategic planning. Its significance lies in its potential to:
\begin{itemize}
    \item Reduce administrative burden through automation
    \item Enhance student performance tracking accuracy with digital records
\end{itemize}

%%%%%%%%%%%%%%%%%%%%%%%%%%%%%%%%%%%%%%%%%%%%%%%%%%%%%%%%%%%%%%
% Chapter 2: Methodology
%%%%%%%%%%%%%%%%%%%%%%%%%%%%%%%%%%%%%%%%%%%%%%%%%%%%%%%%%%%%%%
\chapter{Methodology}
\section{Introduction}
The methodology adopted for this project follows a structured software development lifecycle, encompassing requirements gathering, system analysis, design, implementation, and testing. To ensure the solution was tailored to the specific needs of Susamayawardhana College, a variety of data collection techniques were employed. These included structured interviews with the school principal and administrative staff, questionnaires distributed to teachers to understand their daily challenges, and a thorough review of existing administrative documents, such as student registration forms and attendance logs. This multi-faceted approach ensured a holistic understanding of the operational environment.

\section{Development Approach}
A phased development approach was used:
\begin{itemize}
    \item \textbf{Phase 1: Foundation} -- User authentication, profile management, database setup
    \item \textbf{Phase 2: Core Functionality} -- Academic management, attendance tracking, and assessment features
    \item \textbf{Phase 3: Advanced Features} -- Reporting, communication platform, analytics dashboard
\end{itemize}

\section{Chapter Summary}
This chapter outlined the methodology and development approach used to ensure the system meets stakeholder requirements and is delivered on time.

%%%%%%%%%%%%%%%%%%%%%%%%%%%%%%%%%%%%%%%%%%%%%%%%%%%%%%%%%%%%%%
% Chapter 3: Analysis
%%%%%%%%%%%%%%%%%%%%%%%%%%%%%%%%%%%%%%%%%%%%%%%%%%%%%%%%%%%%%%
\chapter{Analysis}
\section{Current Environment Assessment}
Susamayawardhana College, a provincial school in Colombo 08, operates with limited resources and relies heavily on manual processes, leading to significant inefficiencies and data management challenges. As a school that accommodates students from primary to Advanced Level (A/L), the administrative workload is substantial. The current environment was assessed through stakeholder interviews and process mapping, revealing a need for a digital system to manage student records, academic performance, and parent-teacher communication effectively.

\section{Feasibility Study}
A feasibility study was conducted to evaluate the technical, operational, and economic viability of implementing a school management system at Susamayawardhana College. The results indicated strong potential for improvement through digital transformation, despite the school's resource constraints.

\section{Problem Statement}
At Susamayawardhana College, teachers spend a considerable amount of time on manual administrative tasks, such as marking attendance on paper registers, calculating term-end results, and preparing student reports. This administrative burden detracts from their primary focus on teaching and student development. Meanwhile, parents and guardians struggle to stay informed about their children's academic progress and attendance, often having to wait for parent-teacher meetings for updates. The lack of standardized digital processes results in data inconsistencies and makes it difficult to track long-term student performance data, hindering effective educational planning and intervention.

\section{Chapter Summary}
This chapter analyzed the current system, identified limitations, and established the need for a comprehensive school management solution.

%%%%%%%%%%%%%%%%%%%%%%%%%%%%%%%%%%%%%%%%%%%%%%%%%%%%%%%%%%%%%%
% Chapter 4: Solution Design
%%%%%%%%%%%%%%%%%%%%%%%%%%%%%%%%%%%%%%%%%%%%%%%%%%%%%%%%%%%%%%
\chapter{Solution Design}
\section{System Architecture}
The system is designed with a modular architecture, ensuring scalability and maintainability. Key modules include authentication, academic management, attendance, assessment, reporting, and communication.

\begin{table}[htbp]
    \centering
    \caption{Technologies Used}
    \label{tab:tech-stack}
    \begin{tabular}{ll}
        \toprule
        Category & Technology \\
        \midrule
        Frontend & HTML, CSS, JavaScript \\
        Backend & Node.js, Express.js \\
        Database & MySQL \\
        Version Control & Git \\
        \bottomrule
    \end{tabular}
\end{table}

\section{Design Patterns and Principles}
The design follows best practices such as separation of concerns, robust error handling, and secure data flow. Standardized API interfaces and middleware facilitate integration between modules.

\section{System Modeling and Documentation}
\subsection{Entity-Relationship Diagram}
\begin{figure}[htbp]
    \centering
    \includegraphics[width=1\textwidth]{er-diagram.png}
    \caption{Entity-Relationship Diagram}
    \label{fig:er-diagram}
\end{figure}

\subsection{Class Diagram}
\begin{figure}[htbp]
    \centering
    \includegraphics[width=1\textwidth]{class-diagram.png}
    \caption{Class Diagram}
    \label{fig:class-diagram}
\end{figure}

\subsection{Use Case Diagram}
\begin{figure}[htbp]
    \centering
    \includegraphics[width=1\textwidth]{use-case-diagram.png}
    \caption{Use Case Diagram}
    \label{fig:use-case-diagram}
\end{figure}

\chapter{User Interface Design}
\section{Login Page}
\begin{figure}[htbp]
    \centering
    \includegraphics[width=0.8\textwidth]{login-page.png}
    \caption{Login Page}
    \label{fig:login-page}
\end{figure}

\section{Admin Dashboard}
\begin{figure}[htbp]
    \centering
    \includegraphics[width=1\textwidth]{admin-dashboard-page.png}
    \caption{Admin Dashboard Page}
    \label{fig:admin-dashboard-page}
\end{figure}

\chapter{Sequence Diagrams}
\section{Admin User Management}
\begin{figure}[htbp]
    \centering
    \includegraphics[width=0.8\textwidth]{admin-user-managment-sequence.png}
    \caption{Admin User Management Sequence}
    \label{fig:admin-user-managment-sequence}
\end{figure}

\section{Administrator Report Management}
\begin{figure}[htbp]
    \centering
    \includegraphics[width=1\textwidth]{administrator-report-management-sequence.png}
    \caption{Administrator Report Management Sequence}
    \label{fig:administrator-report-management-sequence}
\end{figure}

\section{Assessment Sequence}
\begin{figure}[htbp]
    \centering
    \includegraphics[width=1\textwidth]{assessment-sequence.png}
    \caption{Assessment Sequence}
    \label{fig:assessment-sequence}
\end{figure}

\section{Attendance Management Sequence}
\begin{figure}[htbp]
    \centering
    \includegraphics[width=1\textwidth]{attendance-management-sequence.png}
    \caption{Attendance Management Sequence}
    \label{fig:attendance-management-sequence}
\end{figure}

\section{Attendance Sequence}
\begin{figure}[htbp]
    \centering
    \includegraphics[width=1\textwidth]{attendance-sequence.png}
    \caption{Attendance Sequence}
    \label{fig:attendance-sequence}
\end{figure}

\section{Authentication Flow Sequence}
\begin{figure}[htbp]
    \centering
    \includegraphics[width=1\textwidth]{authentication-flow-sequence.png}
    \caption{Authentication Flow Sequence}
    \label{fig:authentication-flow-sequence}
\end{figure}

\section{Authentication Sequence}
\begin{figure}[htbp]
    \centering
    \includegraphics[width=1\textwidth]{authentication-sequence.png}
    \caption{Authentication Sequence}
    \label{fig:authentication-sequence}
\end{figure}

\section{Communication Flow Sequence}
\begin{figure}[htbp]
    \centering
    \includegraphics[width=1\textwidth]{communication-flow-sequence.png}
    \caption{Communication Flow Sequence}
    \label{fig:communication-flow-sequence}
\end{figure}

\section{Exam Management Sequence}
\begin{figure}[htbp]
    \centering
    \includegraphics[width=1\textwidth]{exam-management-sequence.png}
    \caption{Exam Management Sequence}
    \label{fig:exam-management-sequence}
\end{figure}

\section{Exam Sequence}
\begin{figure}[htbp]
    \centering
    \includegraphics[width=1\textwidth]{exam-sequence.png}
    \caption{Exam Sequence}
    \label{fig:exam-sequence}
\end{figure}

\section{Generate Monthly Report dministrator Sequence}
\begin{figure}[htbp]
    \centering
    \includegraphics[width=1\textwidth]{generate-monthly-report-dministrator-sequence.png}
    \caption{Generate Monthly Report dministrator Sequence}
    \label{fig:generate-monthly-report-dministrator-sequence}
\end{figure}

\section{Principal Broadcast Sequence}
\begin{figure}[htbp]
    \centering
    \includegraphics[width=1\textwidth]{principal-broadcast-sequence.png}
    \caption{Principal Broadcast Sequence}
    \label{fig:principal-broadcast-sequence}
\end{figure}

\section{Sectional Head Staff Management Sequence}
\begin{figure}[htbp]
    \centering
    \includegraphics[width=0.7\textwidth]{sectional-head-staff-management-sequence.png}
    \caption{Sectional Head Staff Management Sequence}
    \label{fig:sectional-head-staff-management-sequence}
\end{figure}

\section{Student Assesment Management Sequence}
\begin{figure}[htbp]
    \centering
    \includegraphics[width=0.9\textwidth]{student-assesment-management-sequence.png}
    \caption{Student Assesment Management Sequence}
    \label{fig:student-assesment-management-sequence}
\end{figure}

%%%%%%%%%%%%%%%%%%%%%%%%%%%%%%%%%%%%%%%%%%%%%%%%%%%%%%%%%%%%%%
% Chapter 5: Conclusion
%%%%%%%%%%%%%%%%%%%%%%%%%%%%%%%%%%%%%%%%%%%%%%%%%%%%%%%%%%%%%%
\chapter{Conclusion}
The School Management System represents a significant step forward in educational institution management. By addressing current system limitations and incorporating modern technology, the system promises to:
\begin{itemize}
    \item Improve administrative efficiency
    \item Support data-driven decision making
\end{itemize}

The modular design ensures future scalability and adaptability to changing educational needs.

%%%%%%%%%%%%%%%%%%%%%%%%%%%%%%%%%%%%%%%%%%%%%%%%%%%%%%%%%%%%%%
% References
%%%%%%%%%%%%%%%%%%%%%%%%%%%%%%%%%%%%%%%%%%%%%%%%%%%%%%%%%%%%%%
\newpage
\begin{thebibliography}{9}
    \bibitem{ref1} Diploma in Software Engineering Project Guidelines, NIBM
    % Add more references as needed
\end{thebibliography}

%%%%%%%%%%%%%%%%%%%%%%%%%%%%%%%%%%%%%%%%%%%%%%%%%%%%%%%%%%%%%%
% Appendices
%%%%%%%%%%%%%%%%%%%%%%%%%%%%%%%%%%%%%%%%%%%%%%%%%%%%%%%%%%%%%%
\appendix
\chapter{Appendices}
\section{Project Schedule}
The project was executed over a period of six months, following the schedule outlined below.

\begin{longtable}{p{0.2\textwidth} p{0.6\textwidth} p{0.2\textwidth}}
    \caption{Project Execution Timeline}\\
    \toprule
    \textbf{Phase} & \textbf{Key Activities} & \textbf{Duration} \\
    \midrule
    \endfirsthead
    \multicolumn{3}{c}%
    {{\bfseries \tablename\ \thetable{} -- continued from previous page}} \\
    \toprule
    \textbf{Phase} & \textbf{Key Activities} & \textbf{Duration} \\
    \midrule
    \endhead
    \midrule
    \multicolumn{3}{r}{{Continued on next page}} \\
    \midrule
    \endfoot
    \bottomrule
    \endlastfoot
    Phase 1: Planning \& Analysis & Requirements gathering, feasibility study, project planning, initial stakeholder meetings. & 4 Weeks \\
    \midrule
    Phase 2: System Design & System architecture design, database schema design, UI/UX wireframing, technology stack finalization. & 4 Weeks \\
    \midrule
    Phase 3: Core Development & Development of user authentication, profile management, and core academic modules. & 8 Weeks \\
    \midrule
    Phase 4: Feature Implementation & Development of attendance, assessment, and communication features. & 6 Weeks \\
    \midrule
    Phase 5: Testing \& Deployment & Unit testing, integration testing, user acceptance testing (UAT), and initial deployment. & 4 Weeks \\
    \midrule
    Phase 6: Documentation \& Handover & Final report writing, user manual creation, and project handover. & 2 Weeks \\
\end{longtable}

\section{Questionnaires and Interview Questions}
\subsection{For School Administration}
\begin{itemize}
    \item What are the biggest challenges in managing student data currently?
    \item How is communication with parents currently handled?
    \item What are the key reports required by the Ministry of Education?
\end{itemize}

\subsection{For Teachers}
\begin{itemize}
    \item How much time do you spend on administrative tasks daily?
    \item What features in a digital system would be most helpful to you?
    \item How do you currently track student attendance and performance?
\end{itemize}

\section{Meeting Minutes and Log Sheets}
\subsection{Sample Meeting Minutes}
\textbf{Date:} 2024-08-07

\textbf{Attendees:} Project Team, School Vice Principal Seelarathana Thero

\textbf{Agenda:}
\begin{enumerate}
    \item Gether Client requirements
    \item Discussion about the current plan limitations and changes
\end{enumerate}

\textbf{Decisions:}
\begin{itemize}
    \item Changed the teacher permisisons to manage and access student functions to test from student side.
    \item Discussed how to gather user data like fingerprint data and attendance.
\end{itemize}

\section{Reviewed Documents}
The following documents from Susamayawardhana College were reviewed during the analysis phase:
\begin{itemize}
    \item Student Admission Application Form
    \item Manual Attendance Register for Grade 10
    \item Term Test Report Card Template
    \item Staff Leave Application Form
\end{itemize}

\section{Permission Letter}

\begin{figure}[htbp]
    \centering
    \includegraphics[width=0.8\textwidth]{school-permission-lettter.png}
    \caption{School Permission Letter}
    \label{fig:permission-letter}
\end{figure}

%%%%%%%%%%%%%%%%%%%%%%%%%%%%%%%%%%%%%%%%%%%%%%%%%%%%%%%%%%%%%%
% End of Document
%%%%%%%%%%%%%%%%%%%%%%%%%%%%%%%%%%%%%%%%%%%%%%%%%%%%%%%%%%%%%%
\end{document}
